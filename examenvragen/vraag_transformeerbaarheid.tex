\documentclass[examenvragen.tex]{subfiles}

\begin{document}
\section{Tranformeerbaarheid}
\subsection{Opgave}
Wat betekent de volgende zin? ``Probleem $A$ is $\tau(N)$ transformeerbaar tot probleem $B$. Waarvoor kan dergelijke uitspraak nuttig zijn? Geef een voorbeeld van $N$ transformeerbaarheid. 

\subsection{Antwoord}
Als probleem $A$ $\tau(N)$ transformeerbaar is tot probleem $B$ houdt dit het volgende in.
\begin{enumerate}
\item De invoer van probleem $A$ kan omgevormd worden tot geschikte invoer voor probleem $B$.
\item Probleem $B$ kan opgelost worden met een gekend algoritme.
\item Uit de uitvoer voor probleem $B$ kan de oplossing van probleem $A$ afgeleid worden.
\item De eerste en de derde stap samen, kosten $O(\tau(N))$ tijd.
\end{enumerate}
Transformeerbaarheid is nuttig om een ondergrens te vinden voor de complexiteit van een probleem.
Stel bijvoorbeeld dat $A$ $\tau(N)$ transformeerbaar is tot probleem $B$.
\[
A \overset{\tau(N)}{\rightarrow} B
\]
Stel bijvoorbeeld dat je weet dat probleem $A$ minstens $O(R(N))$ tijd vergt, dan houdt dit in dat $O(R(N)-\tau(N))$ een ondergrens is voor de complexiteit van $B$.

Als voorbeeld kiezen we voor probleem $A$ het (vergelijkings)sorteren van $N$ getallen en voor probleem $B$ het vinden van de complex omhullende van $N$ punten. (vergelijkings)Sorteren heeft een ondergrens van $O(N\log N)$. We transformeren het sorteerprobleem als volgt om naar het convex-omhullende probleem. Zij $x_1,\cdots, x_n$ de te sorteren getallen, definieer dan de punten $y_i = (x_i,x_i^2)$ en zoek hiervan de convex omhullende. Loop de convex omhullende af in tegenwijzerzin om de gesorteerde getallen te bekomen. Het sorteren van getallen is dus $N$ transformeerbaar naar het vinden van de convex omhullende. Hieruit volgt dat er een ondergrens is voor de complexiteit van het vinden van een convex omhullende: $O(N\log N - N) = O(N\log N)$.\footnote{Merk op dat tranformeerbaarheid niet symmetrisch is.}
\end{document}
