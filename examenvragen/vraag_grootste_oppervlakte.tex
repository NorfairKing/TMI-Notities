\documentclass[examenvragen.tex]{subfiles}

\begin{document}
\section{Driehoek met de grootste oppervlakte}
\subsection{Opgave}
Gegeven een verzameling punten die op een cirkel liggen. Ontwerp een algoritme om de driehoek (van drie gegeven punten) te vinden met de grootste oppervlakte.

\subsection{Antwoord}
We kunnen dit algoritme eenvoudig ontwerpen met behulp van de theorie over tegenvoetersparen. We weten al dat alle punten op een cirkel tot de convex omhullende behoren van de verzameling punten. Vervolgens kunnen we de grootste driehoek vinden door middel van het algoritme om tegenvoetersparen te vinden. Dit kan zelfs in $O(N)$ tijd!

\end{document}
