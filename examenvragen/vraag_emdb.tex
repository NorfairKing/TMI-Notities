\documentclass[examenvragen.tex]{subfiles}

\begin{document}

\section{EMDB}
\subsection{Opgave}
Wat is een EMDB van een verzameling punten? Wat is het verband tussen de EMDB en het Voronoi diagram van een verzameling punten?
Bespreek beknopt hoe een EMDB van een verzameling kan berekend worden in $O(N\log (N))$ tijd.

\subsection{Antwoord}
EMDB staat voor Euclidisch Minimale DoorloopBoom. Het is een Minimaal opspannende boom gebaseerd op de euclidische afstand als booggewicht.

De EMDB van een verzameling punten is een deelgrafe van de Delaunay triangulatie van die verzameling punten. De Delaunay triangulatie is natuurlijk de duale van het Voronoi diagram.

We kunnen een euclidisch minimaal opspannende boom berekenen in $O(N\log(N))$ tijd met het algoritme van Prim of Kruskal (gekend vanuit fundamenten van de informatica) uitgevoerd op de Delaunay triangulatie.


\end{document}
