\documentclass[examenvragen.tex]{subfiles}

\begin{document}

\section{Voronoi punten en zijden}
\subsection{Opgave}
Bewijs: ``Een voronoi diagramma met $N$ sites heeft maximaal $2N - 5$ voronoipunten en $3N - 6$ voronoi zijden''

\subsection{Antwoord}
\begin{proof}
We bewijzen dit aan de hand van de duaal: de Delaunay-triangulatie.
Een Delaunay-triangulatie is een vlakke verbonden grafe met als knooppunten de Voronoi sites en bogen die de voronoi-zijden `voorstellen'.

In deze triangulatie geldt voor elk knooppunt dat de graad groter is dan, of gelijk aan drie omdat elke Voronoi-veelhoek drie of meer zijden heeft.
Over vlakke grafen weten we een aantal dingen.
\[
V-E+F=2
\]
De som van de graden van alle gebieden is gelijk aan $2E$, en de graad van elk gebied is groter dan of gelijk aan drie.
\[
2E \ge 3F
\]
\[
F = 2-V+E \wedge 2E \ge 3F \Rightarrow E \le 3V+6
\]
Gebruiken we dit nu opnieuw, dan krijgen we de tweede ongelijkheid.
\[
E =V+F-2 \wedge E \le 3V+6 \Rightarrow F\le 2V-4
\]
Deze twee ongelijkheden zijn precies wat we nodig hadden.
\[
E \le 3V-6 \text{ en } F \le 2V-4
\]
Hierin is $V$ het aantal Delaunay knooppunten, of nog, het aantal Voronoi sites, $E$ het aantal Delaunay bogen, of nog, het aantal Voronoi zijden en $F$ het totaal aantal Delaunay gebieden. Bij $F$ wordt ook nog het onbegrensde gebied meegerekend. $F-1$ is dus het aantal Voronoi gebieden. 
\end{proof}

\end{document}
