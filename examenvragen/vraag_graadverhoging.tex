\documentclass[examenvragen.tex]{subfiles}

\begin{document}
\section{Graadverhoging}
\subsection{Opgave}
Bespreek graadverhoding bij B\'ezier-curven, geef het bewijs en leg uit waarvoor het dient.

\subsection{Antwoord}
Elke veelterm van graad $n$ kan geschreven worden als een veelterm van graad $n+1$. Er moet dus ook een manier zijn om een B\'eziercurve van graad $n$ te schrijven als een B\'eziercurve van graad $n+1$.
\[
\sum_{i=0}^{n}b_iB_{i}^{n}(t) = \sum_{i=0}^{n+1}b_i^*B_{i}^{n+1}(t)
\]
De nieuwe controlepunten $b_i^*$ kunnen berekend worden uit de originele controlepunten.
\[
b^*_i= \frac{i}{n+1}b_{i-1} + \left(1-\frac{1}{n+1}\right)b_i
\]
\begin{proof}
Let op: niet vanzelfsprekend.
\[
x(t)
= \sum_{i=0}^{n}b_i\binom{n}{i}(1-t)^{n-i}t^{i}
= \left(\sum_{i=0}^{n}b_i\binom{n}{i}(1-t)^{n-i}t^{i}\right) (t+(1-t))
\]
\[
= \left(\sum_{i=0}^{n}b_i\binom{n}{i}(1-t)^{n-i}t^{i+1}\right)
+ \left(\sum_{i=0}^{n}b_i\binom{n}{i}(1-t)^{n-i+1}t^{i}\right)
\]
Verander de index van de eerste term.
\[
= \left(\sum_{i=1}^{n+1}b_{i-1}\binom{n}{i-1}(1-t)^{n-i+1}t^{i}\right)
+ \left(\sum_{i=0}^{n}b_i\binom{n}{i}(1-t)^{n-i+1}t^{i}\right)
\]
Neem beide sommaties samen. Dit kan omdat we $\binom{n}{-1}$ en $\binom{n}{n+1}$ als nul defini\"eren.
\[
= \sum_{i=0}^{n+1}(1-t)^{n-i+1}t^{i} \left(\binom{n}{i-1}b_{i-1} + \binom{n}{i}b_i\right)
\]
We halen hier de nieuwe controlepunten uit.
\[
\binom{n+1}{i}b_{i}^{*}
= \binom{n}{i-1}b_{i-1}
+ \binom{n}{i}b_i
\]
\[
\frac{(n+1)!}{i!(n-i+1)!}b_{i}^{*}
= \frac{n!}{(i-1)!(n-i+1)!} b_{i-1}
+ \frac{n!}{i!(n-i)!}b_i
\]
\[
b_{i}^{*}
= \frac{i}{n+1} b_{i-1}
+ \left(1-\frac{1}{n+1}\right)b_i
\]
\end{proof}
\noindent Graadverhoging wordt gebruikt om de B\'eziercurve eenvoudig te benaderen omdat de controleveelhoek de curve benadert voor hoge graden.

\end{document}
