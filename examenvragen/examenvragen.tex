\documentclass[12pt,a4paper]{article}
\usepackage[latin1]{inputenc}

\usepackage[dutch]{babel}
\usepackage{listings}

% Voor algoritmes
\usepackage{algorithm2e}

% Voor wiskunde
\usepackage{amsmath}
\usepackage{amsfonts}
\usepackage{amssymb}
\usepackage{amsthm}

% Voor urls
\usepackage{hyperref}

% Voor tekeningen
\usepackage{tikz}

% Om figuren op de juiste plaats te krijgen
\usepackage{float}

% Om de marges aan te passen
\usepackage[left=2cm,right=2cm,top=2cm,bottom=2cm]{geometry}

\usepackage{titlesec}
\usepackage{subfiles}
\usepackage{multicol}
\usepackage{wrapfig}
\usepackage{pdfpages}

% clickable TOC
\usepackage{hyperref}
\hypersetup{
    colorlinks,
    citecolor=black,
    filecolor=black,
    linkcolor=black,
    urlcolor=black
}


\author{Tom Sydney Kerckhove}
\title{Examenvragen TMI}

\usepackage{titlesec}
\newcommand{\partbreak}{\clearpage}
\newcommand{\sectionbreak}{\clearpage}

\renewcommand\thesection{Vraag \arabic{section}}
\renewcommand\thesubsection{V \arabic{section}}

\begin{document}

\pagebreak

\begin{titlepage}
\thispagestyle{empty}
\newcommand{\HRule}{\rule{\linewidth}{0.5mm}}
\center
\textsc{\LARGE KU Leuven}\\[1.5cm]
\vfill


{ \Huge \bfseries Computergestuurd ontwerp van curven en vlakken}\\[0.4cm]
% \HRule \\[1.5cm]
\textsc{\large Toepassingen van de meetkunde in de informatica [G0Q37C]}\\[0.5cm]

\vspace{5cm}

\begin{Large}
Gestart: 21 februari 2014\\
Gecompileerd: \today\\
\end{Large}
\vspace{5cm}

\begin{minipage}{0.4\textwidth}
\begin{flushleft} \large
\emph{Auteur:}\\
Tom Sydney \textsc{Kerckhove}
\end{flushleft}
\end{minipage}
~
\begin{minipage}{0.4\textwidth}
\begin{flushright} \large
\emph{Professor:} \\
Prof. Dr. Ir. Dirk \textsc{Roose}\\
\end{flushright}
\end{minipage}\\[4cm]

\vfill

\end{titlepage}


\iffalse
\part{B\'ezier-curven}
\iffalse
\section{Bespreek $C^{2}$ Continu\"iteit.}
\section{Hoe bekomt men $C^{2}$ continu\"iteit bij samengestelde b\'ezier curven?}
\section{Geef het algoritm van de Casteljau en bewijs de correctheid ervan.}
\section{Bespreek de variatieverminderingseigenschap.}
\section{Wat betekent de variatieverminderingseigenschap voor B\'ezier curven?}
\section{Wat betekent de variatieverminderingseigenschap voor splines?}
\section{Bespreek subdivisie en geef de methode.}
\section{Bespreek graadverhoding, geef het bewijs en leg uit waarvoor het dient.}
\section{Bespreek tensorproductoppervlakken aan de hand van B\'ezier-oppervlakken.}
\section{Bereken de tweede partiele afgeleide in de hoekpunten van een B\'ezier-oppervlak, leg uit en geef grafisch weer.}
\fi

\part{Spline-curven}
\iffalse
\section{Bewijs de eenheidspartitieeigenschap van genormaliseerde B-splines.}
\section{Hoe kan men bij splinecurven punten laten interpoleren door het samennemen van controlepunten? Hoeveel moeten er samenvallen? }
\section{Hoe kan men bij splines punten laten interpoleren door het samennemen van knooppunten?}
\section{Geef het algoritm van de Boor en bewijs de correctheid ervan.}
\section{Hoe kan je ervoor zorgen dat een deel van een spline curve een recht lijnstuk is?}
\fi

\part{Algemeen}
\iffalse
\section{Wat is ``the bounding box quick rejection test'' uit het algoritme om te bepalen of twee rechten elkaar snijden. Geef de twee toepassingen in het algoritme en een implementatie in pseudocode.}https://www.youtube.com/watch?v=HvwMtWkfkJ8 
\section{bespreek het algoritme voor het bepalen van de onderbrug en bovenbrug waarom is de kleinste y-coordinaat een slechte schatting voor het bepalen van de onderbrug?}
\section{geef een algoritme dat in O(N log N) bewerkingen nagaat of in een vz van N lijnstukken er snijdende lijnstukken in voorkomen}
\section{Bewijs correctheid van dit algoritme geef beknopt hoe dit algoritme en de gegevensstrukturen moeten aangepast worden om alle snijdingen te vinden}
\section{geef een algoritme voor de berekening van het VPP van een vz van N punten in O(nlog n) bewerkingen verantwoord de rekencomplexiteit.}
\section{bespreek graham scan + correctheidsbewijs + toon aan dat dit O(nlogn)bewerkingen gebeurt}
\section{wanneer is de inpakmethode (jarvis march) efficienter dan de methode van graham (graham scan)}
\fi

\part{Voronoi diagrammen}
\iffalse
\section{Bewijs: ``Een voronoi diagramma van een verzameling S heeft maximaal $2N - 5$ voronoipunten en $3N - 6$ voronoi zijden''}
\section{Bewijs: ``Een voronoi veelhoek van een punt $p_i$ is begrensd <=> $p_i$ element van $inw(CH(S))$'' Bespreek het nut van deze stelling.}
\section{Bewijs dat minimale doorloopboom deelverzameling is van de Delaunay triangulatie. Wat is het nut van deze eigenschap?}
\section{Bewijs: "Twee dichtste buren hebben een gemeenschappelijke voronoi zijde"}
\section{Geen een strategie en een hoog-niveau algoritme  voor het vinden van de maximale lege cirkel binnen de COV van een verzameling punten.}
\section{Bespreek de event punten in het algoritme van Fortune Welke acties moeten ondernomen worden?}
\fi

\part{Nabijheidsproblemen}
\iffalse
\section{wat betekent volgende uitspraak: probleem A is ?(N) transformeerbaar tot probleem B? waarvoor kan een dergelijke uitspraak nuttig gebruikt worden + vb}
\section{wat is een EMDB van een vz punten + verband met Voronoi diagramma van een vz punten?}
\section{bespreek beknopt hoe een EMDB van een vz punten kan berkend worden in O(nlogn) bewerkingen}
\fi

\part{Padplanning}
\iffalse
\section{Hoe bepaal je het gebied dat kan bereikt worden door een robotarm met 3 segmenten. Is de volgorde van de stukken belangrijk?}
\section{Geef en bespreek de minkovski-som met een voorbeeld.}
\fi
\fi

\part{Oefeningen}
\subfile{vraag_overlappende_veelhoeken}
\subfile{vraag_overlappende_rechthoeken}
\subfile{vraag_snijpunten_van_cirkels}
\subfile{vraag_alle_snijpunten_van_cirkels}
\subfile{vraag_minowski_som}
\subfile{vraag_drie_link_probleem}

\iffalse

\section{Tegenvoetersparen}
gegeven punten op cirkel. Vind de driehoek met het grootste oppervlak.
(opl mbv tegenvoetparen) 


Strategie:
        Doorloop driehoeken met basis 2 opeenvolgende punten; dan met 1 pt. ertussen; ... tot N/2.
        Voor die voorwaarde op de basis, telkens rondlopen gelijkaardig aan de tegenvoetersparen van het verste-puntenpaar-probleem. => Je kan de tegenvoeter van de vorige gebruiken bij de volgende, ... voor dezelfde basis.
        onthoud doorheen het algo de grootste driehoek. 
    Hoog-niveau-algoritme. -> triviaal
    Complexiteit:
        sorteer de punten naar stijgende poolhoek :Nlog(N)
        buitenste lus N/2
        binnenste lus N
        allerbinnenste lus: O(1)
        => totaal O(NlogN + N²) => O(N²) 





\section{Zichtbare lijnstukken}
1 punt gegeven en allemaal lijnstukken (niet snijdende) rond dat punt
schrijf algo om te controleren of je een rechte kunt tekenen vanuit het gegeven punt tot al die rechten
(maw: welke rechten zijn zichtbaar voor dat punt) 

\section{Diagonaal van veelhoek}
gegeven een eenvoudige veelhoek
bepaal of pi, pi+2 een diagonaal is van die veelhoek in O(n) 

\section{Punten in Cirkels}
Gegeven, p cirkels die elkaar niet overlappen, met gegeven middelpunten m en stralen r.
Ook gegeven: n punten p. Elk punt ligt in 1 cirkel.
Bedenk een algoritme dat in O(np lg(np)) kan bepalen welke punten in welke cirkels liggen.
Je moet geen rekening houden met randgevallen, om het algoritme iets simpeler te houden.
Tip: bv doorlooplijnalgoritme 



\fi


\end{document}