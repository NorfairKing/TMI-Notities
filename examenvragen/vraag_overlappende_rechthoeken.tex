\documentclass[examenvragen.tex]{subfiles}

\begin{document}


\section{Overlappende Rechthoeken}
\subsection{Opgave}
Gegeven een verzameling rechthoeken als een koppel van linkeronder-hoekpunt en rechterboven-hoekpunt. Ontwerp een algoritme om alle overlappende rechthoeken te vinden in de verzameling.

\subsection{Antwoord}
De overlappende rechthoek vinden van twee rechthoeken (als deze bestaat) kan in constante tijd en wordt hier niet besproken.
Om alle overlappende rechthoeken te vinden moeten we dus elk paar rechthoeken nakijken die een overlappende rechthoek hebben.

We kunnen dit realiseren in $O((N+S)\log(N))$ tijd met een sweepline algoritme. De linkeronder-punten gebruiken we als toevoegpunten en de rechterboven-punten gebruiken we als verwijderpunten. Ga de event points af van links naar rechts (sorteer ze dus eerst volgens $x$-coordinaat). Vergelijk telkens v\'o\'or het toevoegen van een rechthoek deze rechthoek met alle andere rechthoeken in de status die in hetzelfde interval zitten op de $y$-as. Met behulp van een intervalboom bekomen we op deze manier een complexiteit van $O((N+S)\log(N))$ te bekomen.

\end{document}
