\documentclass[10pt,a4paper]{article}
\usepackage[latin1]{inputenc}
\usepackage[dutch]{babel}
\usepackage{amsmath}
\usepackage{amsfonts}
\usepackage{amssymb}
\usepackage{amsthm}
\usepackage{tikz}
\usepackage{pdfpages}
\title{Oefenzitting 2}
\author{Tom Sydney Kerckhove}
\date{10 maart 2014}

\begin{document}
\maketitle

\noindent\textbf{Spline:}\\
Stuksgewijze veelterm met continu\"iteitsvereisten. Voor een spline van graad $n$ eisen we $C^{n-1}$ continu\"iteit.\\
\noindent\textbf{B-spline:}\\
$n$: graad van veeltermfuncties. $p$: aantal deelintervallen.\\
$p+n$ controlepunten $\vec{b}_i$ met $n+p$ B-splines $N_{i}^{n}(u)$:
$p+2n$ knooppunten.
\[
\vec{s}(u) = \sum_{i=-n}^{p-1}\vec{d}_iN_{i}^{n}(u)
\]

\begin{itemize}
\item Lokaal verschillend van nul:
\[
N_{i}^{n}(u) = 0 \text{ als } u \not\in [u_i, u_{i+n+1})
\]

\item Positiviteit:
\[
N_{i}^{n}(u) \ge 0
\]

\item Eenheidspartitie:
\[
\sum_{i=-n}^{p-1}N_{i}^{n}(u) = 1 \text{ voor } u\in [u_0,u_p)
\]
Wat we doen is dus enigzins zinvol

\item $C_{n-1}$

\end{itemize}

In 2D: $d_{ij}$ controlepunten vormen het controle net.
\[
s(u,v) = \sum_{i=-n}^{p-1}\sum_{i=-m}^{q-1}\vec{d}_{ij}N_{i}^{n}(u)N_{j}^{m}(v)
\]
Tensorproduct: product van meerdere \'e\'endimensionale functies met een verschillende veranderlijke. De uitkomst is dan een tweedimensionale functie.
\[
N_{i,j}^{n,m}(u,v) = N_{i}^{n}(u)N_{j}^{m}(v)
\]
\[
N^{n_1,n_2,...,n_j}_{i_1,i_2,...,i_j}(u_1,u_2,...,u_v) = \prod_{a=1}^jN_{i_a}^{n_a}
\]
.\\\\
Makkelijke evaluatie van meerdere dimensies.
\[
\vec{d_i}^* = \sum_{j=m}^{q-1}\vec{d}_{ij}N_{j}^{m}(u)
\]
\[
s = \sum_{i=-n}^{p-1} \vec{d_{i}}^*N_{i}^{n}(u)
\]
Is gelijk aan
\[
\vec{e_j}^* = \sum_{i=-n}^{p-1}\vec{d}_{ij}N_{i}^{n}(u)
\]
\[
s = \sum_{j=-m}^{q-1} \vec{e_{j}}^*N_{j}^{m}(u)
\]
Maar heeft een verschillende complexiteit.
\[
(n+p)O(q) + O(p) \rightarrow O(p)O(q) + O(p)
\]
tov.
\[
(m+q)O(p) + O(q) \rightarrow O(structureq)O(p) + O(q)
\]structure



Pos maximum: inwendige knooppunten vb: $u_1$ en $u_2$. Neem gemiddelde ervan. $u_1u_2/2$
\section{Vraag 1}
\subsection*{(a)}
\begin{itemize}
\item
VRAGEN AAN HELENA

\end{itemize}
\subsection*{(b)}
\subsection*{(c)}
\subsection*{(d)}
\[
N_{0}^{n}(u) =
\left\{
\begin{array}{r l}
0 &\text{ als } u < u_0\\
  \frac{u-u_0}{u_2-u_0}\frac{u-u_0}{u_1-u_0} &\text{ als } u \in [u_0,u_1)\\
  \frac{u-u_0}{u_2-u_0}\frac{u_2-u}{u_2-u_1} + \frac{u_3-u}{u_3-u_2}\frac{u-u_1}{u_2-u_3} &\text{ als } u \in [u_1,u_2)\\
  \frac{u_3-u}{u_3-u_1}\frac{u_3-u}{u_3-u_2} &\text{ als } u \in [u_2,u_3)\\
0 &\text{ als } u > u_3\\
\end{array}
\right.
\]
\subsection*{(e)}
\begin{proof}
wut...
\end{proof}


wooptiedoo\\


\newcommand\grid{
  % horizontal axis
\draw[->] (0,0) -- (5,0) node[anchor=north] {$u$};
% vertical axis
\draw[->] (0,0) -- (0,4) node[anchor=east] {$N_{i}^{n}(u)$};

% u_i's
\draw	(1,0) node[anchor=north] {$u_0$}
		(2,0) node[anchor=north] {$u_1$}
		(3,0) node[anchor=north] {$u_2$}
		(4,0) node[anchor=north] {$u_3$};

% 
\draw	(0,0)   node[anchor=east] {$0$}
		(0,1.5) node[anchor=east] {$0.5$}
		(0,3)   node[anchor=east] {$1$};
}
\newcommand{\n}[1]{
\draw[thick] (1+#1,3) -- (2+#1,3);
\draw[thick] (0,0) -- (1+#1,0);
\draw[thick] (2+#1,0) -- (5,0);

\node[shape=circle, scale=0.5, fill=black, color=black] at (1+#1,3) {};
\node[shape=circle, scale=0.5, fill=black, color=black] at (2+#1,3) {};
\node[shape=circle, scale=0.5, fill=white, color=black] at (1+#1,0) {};
\node[shape=circle, scale=0.5, fill=white] at (2+#1,0) {};
}
%TODO wtf
\begin{tikzpicture}
\grid
\n{0}
\end{tikzpicture}
\begin{tikzpicture}
\grid
\n{1}
\end{tikzpicture}
\begin{tikzpicture}
\grid
\n{2}
\end{tikzpicture}

\includepdf[pages=-]{opgave.pdf}

\end{document}