\documentclass[10pt,a4paper]{article}
\usepackage[latin1]{inputenc}
\usepackage[dutch]{babel}
\usepackage{amsmath}
\usepackage{amsfonts}
\usepackage{amssymb}
\usepackage{amsthm}
\usepackage{tikz}
\usepackage{pdfpages}
\title{Oefenzitting 3}
\author{Tom Sydney Kerckhove}
\date{24 maart 2014}

\begin{document}
\maketitle


\section*{Oefening 1}
Te bewijzen:
\[
\vec{p_1}\times\vec{p_2}
= \Vert\vec{p_1}\Vert\Vert\vec{p_2}\Vert\sin\alpha
\]
Met de $p$'s voldoende aan:
\[
\vec{p_1} = \begin{pmatrix}
p_{11}\\p_{12}\\0
\end{pmatrix}
\text{ en }
\vec{p_2} = \begin{pmatrix}
p_{21}\\p_{22}\\0
\end{pmatrix}
\]

\begin{proof}
De grootte van het vectorproduct is gelijk aan de oppervlakte van het opgespannen parallellogram. Dit parallellogram heeft zijden $\Vert \vec{p_1}\Vert$ en $\sin\alpha\Vert \vec{p_2}\Vert$.
\end{proof}


\noindent Te bewijzen: Als $\vec{p_1}\times\vec{p_2} <0$  dan ligt $\vec{p_1}$ in tegenwijzerzin ten opzichte van $\vec{p_2}$.
\begin{proof}
Omdat een norm altijd positief is, hangt het teken van $\vec{p_1}\times\vec{p_2}$ af van $\alpha$. Als deze hoek negatief is $\sin\alpha$ negatief, en ligt $\vec{p_1}$ dus in tegenwijzerzin ten opzichte van $\vec{p_2}$.
\end{proof}

\section*{Oefening 2}
Heb het op moeten zoeken:
\[
d = \frac{|(x_0-x_1)\times(x_0-x_2)|}{|x_2-x_1|}
\]

\section*{Oefening 3}
Om de punten te sorteren volgens poolhoek ten opzichte van $\vec{p_0}$ gebruiken we een $O(n\log n)$ sorteeralgoritme met de volgende `kleiner dan' functie.
\[
lt(\vec{p_1},\vec{p_2}) = \vec{p_1}\times\vec{p_2}
\]

In het algemene geval: splits alle punten eerst in de punten boven en onder $p_0$, sorteer ze appart, en voeg ze samen.

\section*{Oefening 4}
Tegenvoorbeeld: neem enkel stijgende poolhoeken zoals een vijfpuntige ster. Alle punten moeten bovendien langs \'e\'en kan van de zijde $p_0p_1$ liggen.

\section*{Oefening 5}
Neem de som van de helft van alle opeenvolgende vectorproducten.
\[
Opp = \sum_{i=0}^{n}\frac{1}{2}\Vert\vec{p}_{i\ mod\ n}\times\vec{p}_{(i+1)\ mod\ n}\Vert
\]
Dit zal just zijn, zelfs voor niet-convexe veelhoeken, omdat het vectorproduct daar niet steeds positief is.


\includepdf[pages=-]{opgave.pdf}

\end{document}